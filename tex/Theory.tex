%! Author = Roy
%! Date = 22.10.2021

% Preamble
\documentclass[twocolumn, sigconf, nonacm, natbib, screen, balance=False]{acmart}

% Packages
\usepackage{clrscode3e}  
\usepackage{listings}
\lstset{language=Python, basicstyle=\ttfamily}

% based on https://tex.stackexchange.com/questions/279240/float-for-lstlisting
\usepackage{float}
\floatstyle{ruled}
\newfloat{listing}{tbph}{lop}
\floatname{listing}{Listing}
\def\lstfloatautorefname{Listing} % needed for hyperref/auroref

\citestyle{acmauthoryear}
% Document
\begin{document}

\section{Theory}\label{sec:theory}

We will be looking at eight different algorithms. Four of which are standard algorithms: Bubble sort, Insertion sort, Quicksort and Mergesort. Then two which are hybrid algorithms: Quicksort with Insertion sort and Mergesort with Insertion sort. Lastly we are going to analyze the built in sorting algoriths in python: Python sort() and Numpy sort().

\subsection{Algo 1: Bubble sort}\label{sec:algo1}

\begin{listing}
  \caption{Insertion sort algorithm from \citet[Ch.~2.1]{CLRS_2009}.}
  \label{lst:bubble_sort}
  \begin{codebox}
    \Procname{$\proc{Bubblesort}(A)$}
    \li \For $i \gets 1$ \To $\attrib{A}{length} - 1$
    \li \Do \For $j \gets \attrib{A}{length}$ \Downto $i + 1$
    \li 	\Do \If $A[j] < A[j - 1]$
    \li         \Then exchange $A[j]$ with $A[j - 1]$
    \End    
    \End
    \End
  \end{codebox}
\end{listing}

Pseudocode for the first algorithm is shown in
Listing~\ref{lst:bubble_sort}. Best case runtime for this algorithm
is

\begin{equation}
  T(n) = \Theta(n) \;.  \label{eq:bub_sort_best}
\end{equation}

It is achieved for correctly sorted input data. Worst case runtime for this algorithm is

\begin{equation}
  T(n) = O(n^2) \;.  \label{eq:bub_sort_worst}
\end{equation}

It is achieve for a reversed sorted input data. Bubblesort is in place and stable.

\subsection{Algo 2: Insertion sort}\label{sec:algo2}

\begin{listing}
  \caption{Insertion sort algorithm from \citet[Ch.~2.1]{CLRS_2009}.}
  \label{lst:insertion_algo}
  \begin{codebox}
    \Procname{$\proc{Insertion-Sort}(A)$}
    \li \For $j \gets 2$ \To $\attrib{A}{length}$
    \li \Do
    $\id{key} \gets A[j]$
    \li     $i \gets j-1$
    \li      \While $i>0$ and $A[i] > \id{key}$
    \li      \Do
    $A[i+1] \gets A[i]$
    \li         $i \gets i-1$
    \End    
    \li       $A[i+1]\gets \id{key}$
    \End
  \end{codebox}
\end{listing}

Insertion sort iterates through the list’s numbers from left to right, moving each number as far to the left as needed for there not to be any lower number to the left of it. It is an in-place algorithm that only stores one auxiliary variable called the key. This conatins the number that is being moved to the left and compared to it’s next.
Pseudocode for the second algorithm is shown in
Listing~\ref{lst:insertion_algo}. Best case runtime for this algorithm
is

\begin{equation}
  T(n) = \Theta(n) \;.  \label{eq:ins_sort_best}
\end{equation}

It is achieved for correctly sorted input data. Worst case runtime for this algorithm is

\begin{equation}
  T(n) = O(n^2) \;.  \label{eq:ins_sort_worst}
\end{equation}

It is achieve for a reversed sorted input data. Insertion sort is in place and stable. 

\subsection{Algo 3: Quicksort}\label{sec:algo3}

\begin{listing}
  \caption{Quicksort algorithm from \citet[Ch.~2.1]{CLRS_2009}.}
  \label{lst:quicksort_algo}
  \begin{codebox}
    \Procname{$\proc{Quicksort}(A, p, r)$}
    \li \If $p < r$
    \li \Then $q \gets \proc{Partition}(A, p, r)$
    \li     $\proc{Quicksort}(A, p, q - 1)$
    \li     $\proc{Quicksort}(A, q + 1, r)$
    \End
  \end{codebox}
\end{listing}

\begin{listing}
  \caption{Partition from \citet[Ch.~2.1]{CLRS_2009}.}
  \label{lst:partition}
  \begin{codebox}
    \Procname{$\proc{Partition}(A, p, r)$}
    \li $x \gets A[r]$
    \li $i = p - 1$
    \li \For $j \gets p$ \To $r - 1$
    \li \Do \If $A[j] \le x$
    \li  	\Then $i \gets i + 1$
    \li     exchange $A[i]$ with $A[j]$
    \End
    \End
    \li exchange $A[i + 1]$ with $A[r]$
    \li \Return $i + 1$
  \end{codebox}
\end{listing}

Quicksort is a “divide and conquer” algorithm.
Pseudocode for the third algorithm is shown in
Listing~\ref{lst:quicksort_algo} and~\ref{lst:partition}. Best case runtime for this algorithm is

\begin{equation}
  T(n) = \Theta(n*log(n)) \;.  \label{eq:quick_sort_best}
\end{equation}

It is achieved for where it chooses pivot always in the middle and average case for when the array is random. Worst case runtime for this algorithm is

\begin{equation}
  T(n) = O(n^2) \;.  \label{eq:quick_sort_worst}
\end{equation}

It is achieved for when the input data is both reversed sorted and sorted.

\subsection{Algo 4: Quicksort Insertion sort hybrid}\label{sec:algo5}

\begin{listing}
  \caption{Quicksort insertion sort hybrid from GeeksforGeeks advanced algorithm}
  \label{lst:quickinsert_algo}
  \begin{codebox}
    \Procname{$\proc{QuickInsertion-Sort}(A, p, r)$}
    \li \While $p<r$
    \li	\Do \If $r-p+1<10$
    \li		\Do $\proc{Insertion-Sort(A})$
    \li			End loop
    \li 	\Else
    \li			$x \gets \proc{Partition}(A, p, r)$
    \li 	\If $x-p<r-x$
    \li		\Do $\proc{QuickInsertion-Sort}(A, p, x-1)$
    \li 	$p \gets x+1$
    \li 	\Else
    \li 	$\proc{QuickInsertion-Sort}(A, x+1, r)$
    \li 	$r \gets x-1$
    \End
    \End
    \End
  \end{codebox}
\end{listing}

Pseudocode for the fourth algorithm is shown in
Listing~\ref{lst:quickinsert_algo}. Best case runtime for this algorithm
is

\begin{equation}
  T(n) = \Theta(n) \;.  \label{eq:quick_sort_best}
\end{equation}

It is achieved for correctly sorted input data.

\subsection{Algo 5: Mergesort}\label{sec:algo6}

\begin{listing}[H]
  \caption{Mergesort algorithm from \citet[Ch.~2.1]{CLRS_2009}.}
  \label{lst:mergesort_algo}
  \begin{codebox}
    \Procname{$\proc{Merge-Sort}(A, p, r)$}
	\li \If $p < r$
	\li \Then $q \gets \left\lfloor(p + r) / 2\right\rfloor$
	\li 	$\proc{Merge-Sort}(A, p, q)$
	\li 	$\proc{Merge-Sort}(A, q+1, r)$
	\li 	$\proc{Merge}(A, p, q, r)$
	\End
  \end{codebox}
\end{listing}

\begin{listing}[H]
  \caption{Merge from \citet[Ch.~2.1]{CLRS_2009}.}
  \label{lst:merge}
  \begin{codebox}
    \Procname{$\proc{Merge}(A, p, q, r)$}
	\li $n_1 \gets q-p+1$
	\li $n_2 \gets r-q$
	\li let $L[1..n_1+1]$ and $R[1..n_2+1]$ be new arrays
	\li \For $i \gets 1$ \To $n_1$
	\li \Do $L[i] \gets A[p+i-1]$
	\End
	\li \For $j \gets 1$ \To $n_2$
	\li \Do $R[j] \gets A[q+j]$
	\End
	\li $L[n_1+1] \gets \infty$
	\li $R[n_2+1] \gets \infty$
	\li $i \gets 1$
	\li $j \gets 1$
	\li \For $k \gets p$ \To $r$
	\li \Do \If $L[i] \le R[j]$
	\li 	\Then $A[k] \gets L[i]$
	\li 		$i \gets i+1$
	\li 	\Else $A[k] \gets R[j]$
	\li 	$j \gets j+1$
	\End
	\End
  \end{codebox}
\end{listing}

Pseudocode for the fifth algorithm is shown in
Listing~\ref{lst:merge}. The runtime for this algorithm in all cases is

\begin{equation}
  T(n) = \Theta(n*log(n)) \;.  \label{eq:ins_sort_best}
\end{equation}

\subsection{Algo 6: Mergesort Insertion sort hybrid}\label{sec:algo7}

\begin{listing}
  \caption{Mergesort Insertion sort hybrid from GeeksforGeeks timsort}
  \label{lst:mergeinsert_algo}
  \begin{codebox}
    \Procname{$\proc{MergeInsertion-Sort}(A)$}
    \li $minrun \gets 32$
    \li \For $s \gets 0$ \To $\attrib{A}{length}$ \By $minrun$
    \li \Do $e \gets$ lesser~of $s+minrun-1$ and $\attrib{A}{length}-1$
    \li		$\proc{Insertion-Sort}(A, s, e)$
    \End
    \li $size \gets minrun$
    \li \While $h<\attrib{A}{length}$
    \li \Do \For $p \gets 0$ \To $\attrib{A}{length}$ \By $2*size$
    \li 	\Do $q \gets$ lesser~of $\attrib{A}{length}-1$ and $p+size-1$
    \li			$r \gets$ lesser~of $p+2*size-1$ and $\attrib{A}{length}-1$
    \li			\If $q<r$
    \li 		\Do $\proc{Merge}(A, p, q, r)$
    \End
    \End
    \li 	$size \gets 2*size$
    \End		
  \end{codebox}
\end{listing}

\begin{listing}
  \caption{Insertion sort from GeeksforGeeks timsort}
  \label{lst:insert_algo}
  \begin{codebox}
    \Procname{$\proc{Insertion-Sort}(A, p, r)$}
    \li \For $i \gets p+1$ \To $r+1$
    \li \Do $j \gets i$
    \li 	\While $j>p$ and $A[j]<A[j-1]$
    \li		\Do exchange $A[j]$ with $A[j-1]$
    \End
    \End		
  \end{codebox}
\end{listing}

Pseudocode for the second algorithm is shown in
Listing~\ref{lst:mergeinsert_algo}. Best case runtime for this algorithm
is

\begin{equation}
  T(n) = \Theta(n) \;.  \label{eq:ins_sort_best}
\end{equation}

It is achieved for correctly sorted input data.

\subsection{Algo 7: Python sort}\label{sec:algo8}

Built in sorting algorithm standard in the python programming language

\subsection{Algo 8: Numpy sort}\label{sec:algo9}

Built in sorting algortihm from Numpy a python package.

\end{document}
